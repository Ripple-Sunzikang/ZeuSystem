\documentclass[UTF8,a4paper,12pt]{ctexrep}
\usepackage{geometry}
\geometry{left=2.5cm,right=2.5cm,top=2.5cm,bottom=2.5cm}
\usepackage{graphicx}
\usepackage{float}
\usepackage{listings}
\usepackage{xcolor}
\usepackage{booktabs}
\usepackage{longtable}
\usepackage{array}
\usepackage{hyperref}
\usepackage{amsmath}
\usepackage{caption}
\usepackage{fancyhdr}
\usepackage{enumitem}

\hypersetup{colorlinks=true,linkcolor=black,urlcolor=blue,citecolor=black}
\linespread{1.2}
\setlength{\parindent}{2em}
\setlength{\parskip}{0.3em}

\pagestyle{fancy}
\fancyhf{}
\fancyhead[L]{\leftmark}
\fancyhead[R]{SEU-RISCV-CPU}
\fancyfoot[C]{\thepage}
\renewcommand{\headrulewidth}{0.4pt}
\setlength{\headheight}{15pt}

\definecolor{codegray}{rgb}{0.5,0.5,0.5}
\definecolor{codepurple}{rgb}{0.58,0,0.82}
\definecolor{codeblue}{rgb}{0,0,0.6}

\lstdefinelanguage{Rust}{
  morekeywords={as,break,const,continue,crate,else,enum,extern,false,fn,for,if,impl,in,let,loop,match,mod,move,mut,pub,ref,return,self,Self,static,struct,super,trait,true,type,unsafe,use,where,while},
  sensitive=true,
  morecomment=[l]{//},
  morecomment=[s]{/*}{*/},
  morestring=[b]",
}

\lstdefinestyle{codestyle}{
  backgroundcolor=\color{white},
  commentstyle=\color{codegray},
  keywordstyle=\color{codeblue},
  stringstyle=\color{codepurple},
  basicstyle=\ttfamily\small,
  breaklines=true,
  captionpos=b,
  keepspaces=true,
  numbers=left,
  numbersep=6pt,
  numberstyle=\tiny\color{codegray},
  showstringspaces=false,
  tabsize=4,
  frame=single
}
\lstset{style=codestyle}

\newcommand{\frontmatter}{\pagenumbering{roman}}
\newcommand{\mainmatter}{\cleardoublepage\pagenumbering{arabic}}
\newcommand{\backmatter}{\cleardoublepage}

\newcommand{\figplaceholder}[2][0.28\textheight]{%
\begin{figure}[H]
\centering
\fbox{\parbox[c][#1][c]{0.9\textwidth}{
\centering \textbf{#2}\\[0.6em]
(此处放置图示,占位)
}}
\caption{#2}
\end{figure}
}

\begin{document}

\begin{titlepage}
\centering
\vspace*{2.5cm}
{\Huge\bfseries SEU-RISCV-CPU 项目汇报\par}
\vspace{0.8cm}
{\Large 从软件到硬件的完整实现\par}
\vspace{2.5cm}
\begin{tabular}{rl}
项目名称: & SEU-RISCV-CPU \\
作者: & \underline{\hspace{7cm}} \\
学号: & \underline{\hspace{7cm}} \\
指导老师: & \underline{\hspace{7cm}} \\
学院/专业: & \underline{\hspace{7cm}} \\
\end{tabular}
\vfill
{\large \today\par}
\end{titlepage}

\frontmatter
\chapter*{摘要}
本项目实现了一个从软件到硬件的完整 RISC-V 教学型 CPU 系统,覆盖 C 语言子集编译器、BIOS 固件、五级流水线 CPU 核心、SoC 外设与 FPGA 工程。系统支持基本的算术逻辑运算、分支跳转和字长访存,具备 4×4 键盘、数码管、LED、UART、定时器、PWM、看门狗等外设控制能力,展示了从源代码、汇编到 COE 初始化文件,再到 FPGA 上运行的完整链路。本报告围绕架构设计、指令集支持、编译器与 BIOS 实现、硬件微架构、冒险处理、存储映射、外设控制以及 FPGA 工程流程展开,并给出关键代码清单与图例占位,为后续复现与扩展提供系统化文档。

\textbf{关键词}:RISC-V;流水线;编译器;BIOS;FPGA;SoC

\chapter*{Abstract}
This project delivers a full-stack RISC-V educational CPU system, spanning a C-subset compiler, BIOS firmware, a five-stage pipelined core, SoC peripherals, and an FPGA implementation flow. The system supports basic arithmetic/logic, branches, and word-level memory access, and integrates keypad, seven-seg display, LEDs, UART, timers, PWM, and watchdog. It demonstrates the end-to-end path from C source code to assembly, COE initialization, and FPGA execution. This report details the architecture, ISA subset, compiler/BIOS implementation, microarchitecture, hazard handling, memory mapping, peripheral controllers, and FPGA flow, with code listings and figure placeholders for future refinement.

\textbf{Keywords}: RISC-V, pipeline, compiler, BIOS, FPGA, SoC

\tableofcontents
\listoffigures
\listoftables

\mainmatter

\chapter{项目背景与目标}
\section{背景与意义}
RISC-V 作为开源指令集架构,具备开放、可裁剪和生态友好的特点,非常适合作为教学与研究平台。为了让学生从软件到硬件完整理解处理器系统,本项目实现了一个“可运行、可编译、可上板”的最小可用系统,涵盖编译器、运行时固件、CPU 核心与外围设备,实现了从 C 代码到 FPGA 运行的完整链路。

传统教学往往在“软件”和“硬件”之间存在较大鸿沟:软件课程侧重语言与算法,硬件课程侧重电路与时序。一个完整的端到端系统可以把这两部分连接起来,使学生真正理解“高级语言如何变成电路上的信号变化”。这正是本项目的重要意义所在。

此外,RISC-V 的开放性使得学生可以自由查看并修改指令集与实现细节,从而形成更深入的理解,而不是停留在黑盒层面。

\section{项目目标与交付物}
本项目的目标不仅是实现一个能执行指令的 CPU 核心,更强调全栈链路的完整性。项目交付物包括:
\begin{enumerate}[label=\arabic*.]
\item C 语言子集编译器(Rust 实现),支持词法/语法/语义分析与代码生成;
\item BIOS 固件与系统服务层,封装显示、输入、串口、计时等功能;
\item RV32I 子集五级流水线 CPU 核心与 SoC;
\item 外设控制器与 MMIO 地址映射;
\item Vivado FPGA 工程与下载流程;
\item 示例应用(计算器、LED 波浪灯、拨码开关演示)。
\end{enumerate}

从教学角度看,交付物的价值不仅在于“结果可用”,更在于“过程可学”:每个子模块都可以独立讲解与实验验证,组合在一起则形成完整系统。

\section{设计原则与评价指标}
本项目强调“可解释、可复现、可上板”的教学价值,因此在体系结构选择与工程实现上遵循以下原则:其一,结构清晰,模块边界明确,便于课堂讲解与后续扩展;其二,软硬件协同,编译器、BIOS 与硬件接口一一对应,避免“纸上架构”;其三,实验可操作,通过有限的外设完成典型输入输出任务。

评价指标方面,除了功能正确性外,还关注以下维度:流水线是否稳定、外设访问是否一致、编译器生成的代码是否可执行、FPGA 时序与资源是否满足约束。换言之,本项目既是“能跑起来的 CPU”,也是“能讲清楚的系统”。

\section{设计约束与取舍}
为了保证教学可理解性与可上板性,本项目做了若干取舍:
\begin{itemize}
\item 指令集为 RV32I 子集,仅包含 add/sub、逻辑运算、移位、基本分支、jal/jalr 与 lw/sw;
\item 采用五级流水线结构(IF/ID/EX/MEM/WB),不引入缓存与乱序执行;
\item 外设通过 MMIO 访问,以轮询为主,不依赖中断;
\item 编译器支持 C 子集,避免复杂语义(结构体、浮点等)。
\end{itemize}

这些约束直接影响了系统复杂度与实现成本。例如,去除乘除与浮点可显著简化 ALU 与编译器;采用轮询方式可以避免中断控制器设计,使 BIOS 更加直观;而五级流水线则在性能与可解释性之间取得平衡,既能展示流水线概念,又不至于引入过多乱序与缓存细节。

\figplaceholder{项目全栈流程示意图}

\chapter{系统总体架构与流程}
\section{总体架构}
系统分为应用层、BIOS 服务层和硬件层三部分。应用层只关注业务逻辑,通过 BIOS 提供的函数访问硬件;BIOS 通过 MMIO 与硬件交互;硬件层包含 CPU 核心、存储器与外设控制器,组成一个小型 SoC。

从信息流角度看,应用层产生“需求”,BIOS 将其转换为“标准化的硬件访问”,硬件层只需保证 MMIO 读写语义正确即可。这样的分层不仅降低应用编写门槛,也让硬件在不修改应用的前提下可持续扩展。对于教学而言,学生可以分别理解“软件调用接口”“接口映射到寄存器”“寄存器驱动硬件”的完整链路。

从控制流角度看,CPU 负责执行指令序列,BIOS 则提供一组最小系统调用,使得应用逻辑不需要直接操作地址与位宽细节。该模式类似于“微型操作系统 + 裸机驱动”的混合模型,既保留了裸机可控性,又提供了友好的编程抽象。

\figplaceholder{miniRV SoC 顶层结构示意图}

\section{目录结构}
\subsection{目录树(节选)}
\noindent 为了便于上手与定位问题,仓库目录遵循“工具链—示例—硬件工程”三大块的组织方式。编译器源码集中在 \texttt{src/},硬件工程位于 \texttt{rvTest/},示例程序在 \texttt{examples/}。输出产物(汇编、COE)统一放入 \texttt{output/},方便与 Vivado 工程衔接。

\noindent 对于第一次接触的读者,建议从 \texttt{QuickStart.md} 入手完成上板流程,再回到 \texttt{README.md} 阅读架构说明,最后进入 \texttt{src/} 与 \texttt{rvTest/} 深入细节。
\begin{lstlisting}[caption={项目目录树(节选)}]
SEU-RISCV-CPU/
├── Cargo.toml
├── README.md
├── QuickStart.md
├── build.sh
├── src/
│   ├── main.rs
│   ├── lexer.rs
│   ├── parser.rs
│   ├── codegen.rs
│   ├── assembler/
│   └── linker/
├── examples/
│   ├── bios_v2.c
│   ├── calculator_v2.c
│   ├── sw_led_demo.c
│   └── led_wave.c
├── output/
│   ├── calc_v2.s
│   └── calc_v2.coe
├── rvTest/
│   └── rvTest.srcs/
└── zeus_ide/
\end{lstlisting}

\begin{longtable}{p{0.28\textwidth} p{0.68\textwidth}}
\caption{项目目录结构与说明}\label{tab:dir}\\
\toprule
路径 & 说明 \\
\midrule
\endfirsthead
\toprule
路径 & 说明 \\
\midrule
\endhead
\midrule
\multicolumn{2}{r}{续下页}\\
\endfoot
\bottomrule
\endlastfoot
\texttt{src/} & 编译器核心代码,包含词法、语法、语义分析、代码生成、汇编与链接。\\
\texttt{examples/} & BIOS 与应用示例源代码(计算器、LED、拨码开关等)。\\
\texttt{output/} & 编译生成的汇编和 COE 文件示例。\\
\texttt{rvTest/} & Vivado 硬件工程与 RTL 源码(CPU 核心、外设、顶层)。\\
\texttt{zeus\_ide/} & 可选的编译器 IDE(Electron)。\\
\texttt{QuickStart.md} & 从 C 到 FPGA 的快速上手指南。\\
\texttt{README.md} & 详细项目文档与架构说明。\\
\end{longtable}

\section{从 C 到 FPGA 的流程}
编译与部署遵循以下流程。该流程体现了“从高级语言到硬件执行”的完整链路,每一步都会产生可验证的中间结果(汇编、COE、比特流),便于定位问题。
\begin{enumerate}[label=\arabic*.]
\item 编译 Rust 编译器;
\item 将 BIOS 与应用程序链接,生成 RISC-V 汇编;
\item 汇编与链接生成 COE 初始化文件;
\item 将 COE 文件拷贝至 Vivado 工程;
\item 综合、实现并生成比特流;
\item 下载至 FPGA 开发板运行。
\end{enumerate}

\noindent 其中,BIOS 与应用在源级链接,保证统一的入口与调用约定;而 COE 文件则是 Vivado 初始化 IROM/PRAM 的标准格式,使软件内容以“硬件可读”的方式进入 FPGA。

\begin{lstlisting}[language=bash,caption={编译器与示例程序的基本使用}]
# 编译编译器
cargo build --release

# 编译 BIOS + 应用,生成汇编与 COE
./target/release/riscv_compiler examples/bios_v2.c examples/calculator_v2.c -o output/calc_v2

# 拷贝 COE 到 Vivado 工程
cp output/calc_v2.coe rvTest/rvTest.ip_user_files/mem_init_files/program.coe
\end{lstlisting}

\noindent 为保证流程可靠,建议在每一步进行最小验证:检查汇编是否生成、COE 条目数是否符合预期、Vivado 是否成功读取初始化文件,从而缩小错误定位范围。

\figplaceholder{软件到硬件的流水线流程图}

\chapter{指令集与执行模型}
\section{RV32I 子集支持}
CPU 核心实现了 RV32I 的一个可运行子集。指令选择以教学与实验为目标,覆盖算术逻辑、立即数、分支跳转与字长访存。编译器保证只产生硬件支持的指令。

子集的选择遵循“最小可用”原则:优先支持能够构成控制流与数据流的指令组合,使得常见的 C 语句(赋值、分支、循环、函数调用)能够被正确翻译与执行。这样即使指令数量不多,也可以覆盖较完整的编程实践。

\begin{table}[H]
\centering
\caption{支持的指令子集(与控制器实现保持一致)}
\begin{tabular}{ll}
\toprule
类别 & 指令 \\
\midrule
算术逻辑(R 型) & add, sub, and, or, xor, sll, srl, sra \\
立即数运算(I 型) & addi, andi, ori, xori, slli, srli, srai \\
访存 & lw, sw \\
分支跳转 & beq, bne, blt, bge, jal, jalr \\
其他 & lui \\
\bottomrule
\end{tabular}
\end{table}

\section{指令格式与寄存器模型}
指令遵循 R/I/S/B/U/J 六种基本格式,立即数在译码阶段进行符号扩展。寄存器文件为 32 个 32 位通用寄存器(x0 固定为 0),以同步写入、组合读取的方式实现。

从硬件实现角度看,指令格式的核心差异集中在立即数的拼接方式与寄存器字段的位置。通过统一的译码逻辑与符号扩展模块(SEXT),可以将不同格式的立即数转化为 32 位有符号数,供 ALU 或 NPC 使用。寄存器模型遵循 RISC-V 约定:x0 永远为 0,有利于零常数使用与简化硬件。

在调用约定上,系统采用简化的寄存器传参与栈帧机制,保证函数调用与返回路径清晰可控。这一设计同时考虑了教学可解释性与编译器实现难度。

\figplaceholder{R/I/S/B/U/J 指令格式示意图}

\section{执行模型与内存系统}
处理器采用小端序,字长为 32 位。内存访问以字为单位,lw/sw 为核心数据访问指令。MMIO 地址空间映射外设寄存器,BIOS 通过内存读写实现硬件访问。

需要强调的是,本设计并未实现字节/半字访存,因此编译器在生成访存指令时以 32 位对齐为基本假设。对齐策略简化了存储器接口与数据通路,但也限制了可支持的数据结构类型。这样的取舍有助于教学理解,同时避免引入复杂的字节选通与对齐异常逻辑。

MMIO 的设计使得外设与内存共享统一的访问语义:软件层面只需读写地址,硬件层面通过 Bridge 完成译码与路由。这样既能保留“内存读写即外设控制”的直观性,又能保持 CPU 核心对外设的透明性。

\section{汇编示例}
下面给出编译器生成的 RISC-V 汇编片段,用于说明启动初始化与外设映射访问的基本结构(示意,不完整):
\begin{lstlisting}[language={[x86masm]Assembler},caption={汇编片段示例(示意)}]
.text
_start:
    lui sp, 0x00008         # 初始化栈指针
    addi sp, sp, -4
    lui t0, 0xFFFFF         # 外设基址
    addi t0, t0, -1024
    sw   a0, 0(t0)           # 向外设写入
    lw   a1, 0(t0)           # 从外设读取
    jal  ra, user_main
\end{lstlisting}

该片段展示了两个关键点:一是启动阶段需要初始化栈指针,以保证函数调用可用;二是外设访问仅依赖内存读写指令,通过外设基址完成读写操作。应用层与 BIOS 通过这种机制完成“软件到硬件”的桥接。

\chapter{编译器设计与实现}
\section{总体架构}
编译器以 Rust 实现,流程为:词法分析→语法分析→语义分析→代码生成→汇编→链接→COE 输出。整体目标是将 C 子集转换为 RV32I 汇编,并完成简化的链接与内存布局。

这一流程被刻意设计为“透明可观测”:每一阶段都有可输出的中间结果(Token、AST、汇编文本),便于教学演示与调试。相比于黑盒式的编译器,本项目更强调可追踪性与可解释性。

\figplaceholder{编译器前端与后端流程示意图}

\section{词法与语法分析}
词法分析将源码切分为 Token;语法分析构建 AST,并对表达式、语句与函数结构进行解析。为保证可用性,语法覆盖 if/while/for、函数调用与基本表达式。
\noindent 设计重点在于“够用且清晰”:只解析项目所需的语法结构,同时保证错误提示尽量指向具体位置。对于教学场景来说,清晰的语法错误比复杂语法覆盖更重要。

\section{语义分析与代码生成}
语义分析负责类型检查、符号表维护与作用域管理。代码生成根据 AST 输出 RV32I 指令序列,并处理函数调用约定、寄存器分配与简单栈帧布局。
\noindent 代码生成的关键是“把高级语义映射到有限指令集”。例如,乘法在硬件中未直接支持时,编译器通过调用 BIOS 软件乘法例程完成;数组与指针访问在生成阶段被转化为基址加偏移的访存指令。栈帧采用固定大小分配,降低了寄存器分配与栈布局的实现复杂度。

\noindent 同时,为适配硬件的存储器时序,编译器在必要位置插入空操作(如 store 后的 load),避免出现简单的访存冒险。这体现了软硬协同的设计思想:硬件简化,软件补偿。

\section{汇编器与链接器}
汇编器负责将汇编文本转换为 ELF 结构;链接器根据段布局对符号进行重定位,并输出最终的 COE 文件。链接器将 .text 段放置在 0x0000\_0000 起始地址,并确保数据段不与 MMIO 冲突。
\noindent 由于目标环境是 FPGA 上的初始化存储器,链接器在布局阶段以“简单可控”为目标:段连续排列、按字对齐、地址范围可预测。这样既便于调试,也方便教学解释内存布局。

\figplaceholder{COE 输出格式示意图}

\chapter{BIOS 固件与软件运行时}
\section{BIOS 设计目标}
BIOS 负责提供硬件抽象层,将 MMIO 细节封装为简单函数接口,应用程序仅需调用 BIOS 函数即可实现显示、输入、串口通信与外设控制。

BIOS 的核心价值在于“统一接口”。对于应用开发者而言,只需理解函数语义,而不必关心地址映射与寄存器位定义;对于硬件开发者而言,只需保证寄存器语义一致即可。这种分离显著降低了应用编写门槛,也提高了系统可维护性。

\section{启动流程与 Bootloader}
系统上电后,BIOS 负责初始化栈指针、外设自检,并根据拨码开关选择进入用户程序或 UART Bootloader。Bootloader 允许通过串口下载新程序并写入指定存储区域。

启动流程中包含的外设自检(如 LED/数码管点亮与键盘检测)不仅用于确认硬件连接,也为后续实验提供可靠的“已知正常”基线。Bootloader 的存在使得程序更新无需重新综合工程,提升了迭代效率。

\figplaceholder{BIOS 启动流程示意图}

\section{BIOS 接口列表(节选)}
\noindent BIOS 接口可按功能划分为显示类(数码管)、输入类(键盘、按钮、拨码开关)、通信类(UART)、音频类(PWM/蜂鸣器)以及系统维护类(看门狗)。接口数量并不追求全面,而是覆盖教学实验常见需求。
\begin{longtable}{p{0.35\textwidth} p{0.15\textwidth} p{0.15\textwidth} p{0.27\textwidth}}
\caption{BIOS 接口概要}\label{tab:bios}\\
\toprule
函数 & 参数 & 返回值 & 说明 \\
\midrule
\endfirsthead
\toprule
函数 & 参数 & 返回值 & 说明 \\
\midrule
\endhead
\midrule
\multicolumn{4}{r}{续下页}\\
\endfoot
\bottomrule
\endlastfoot
\texttt{bios\_display\_bcd} & int & void & 数码管显示(支持负数)\\
\texttt{bios\_key\_read} & - & int & 读取键盘(无按键返回 -1)\\
\texttt{bios\_led\_write} & int & void & 写 24 位 LED \\
\texttt{bios\_uart\_putc} & char & void & 发送单字符 \\
\texttt{bios\_uart\_puts} & char* & void & 发送字符串 \\
\texttt{bios\_uart\_getc} & - & char & 阻塞接收字符 \\
\texttt{bios\_buzzer\_set} & int & void & 设置 PWM 频率并开启 \\
\texttt{bios\_sw\_read} & - & int & 读取拨码开关状态 \\
\texttt{bios\_btn\_read} & - & int & 读取按钮状态 \\
\texttt{bios\_wdt\_feed} & - & void & 看门狗喂狗 \\
\end{longtable}

\section{应用示例:计算器}
应用程序只需实现 \texttt{main()},通过 BIOS 函数完成输入与显示,逻辑集中在按键解析与算术运算。

计算器示例体现了“输入—处理—输出”的典型嵌入式闭环:输入来自 4×4 键盘,处理在 CPU 内完成,输出显示在数码管。该示例既能验证键盘扫描与数码管显示功能,也能验证基础算术运算的正确性。

\begin{lstlisting}[language=C,caption={计算器应用片段(calculator\_v2.c)}]
if (key == 10) { op = 1; input_mode = 2; }
if (key == 11) { op = 2; input_mode = 2; }
if (key == 12) { op = 3; input_mode = 2; }

if (key == 13) {
    if (op == 1) { result = num1 + num2; }
    if (op == 2) { result = num1 - num2; }
    if (op == 3) { result = bios_multiply(num1, num2); }
    bios_display_bcd(result);
}
\end{lstlisting}

\chapter{软硬件协同设计}
\section{协同设计目标}
本项目的协同设计目标是“硬件尽量简单,软件适度补偿”。硬件侧保持指令集与数据通路的最小可用,实现清晰可讲的微架构;软件侧通过 BIOS 与编译器完成必要的功能补足,例如软件乘法、外设抽象与必要的延迟插入。

这种协同思路可以显著降低硬件实现复杂度,同时保证应用层仍能使用接近 C 的开发方式。对教学而言,它帮助学生理解“硬件能力有限时,软件如何承担补偿责任”。

\section{调用约定与栈帧模型}
调用约定采用简化版:参数优先使用寄存器传递,返回值保存在 \texttt{a0},返回地址保存在 \texttt{ra}。为保持行为可预测,函数进入时分配固定大小栈帧并保存返回地址,函数退出时恢复。

固定栈帧虽然牺牲了一定的空间效率,但换来更清晰的控制流与更易观察的栈指针变化,便于实验与调试。

\section{BIOS 系统调用表机制}
BIOS 提供系统调用表,编译器在生成对 BIOS 函数的调用时,会将函数名映射为表项索引,再通过间接跳转完成调用。这样可以避免在应用代码中硬编码具体的 BIOS 地址,提升可维护性与扩展性。

从原理上看,这一机制类似于“动态链接表”,将应用与 BIOS 解耦,使得 BIOS 版本迭代时应用无需重新适配地址映射。

\section{内存布局与软件补偿}
编译器与链接器共同决定内存布局:\texttt{.text} 段从固定地址开始连续摆放,数据段与栈段保持明确边界,避免与 MMIO 地址冲突。由于硬件未实现完整的访存冒险处理,编译器在必要位置插入空操作,保证访存时序安全。

\figplaceholder{软硬件协同调用链示意图}

\chapter{CPU 微架构设计}
\section{设计总览}
本 CPU 以 RV32I 子集为目标,采用五级流水线结构,强调“结构清晰、信号明确、可教学复现”。微架构的核心思想是将指令执行过程拆分为若干阶段,每个阶段只完成相对单一的职责,从而降低组合逻辑复杂度并提高整体吞吐率。

为了保证可解释性,本设计避免引入缓存、乱序执行与分支预测等复杂机制。这样做的代价是性能上不追求极限,但可以换来“每条指令的执行路径都可被完整描述”的教学价值。

\section{数据通路与控制通路分离}
数据通路负责“数据的流动与运算”,控制通路负责“路径选择与时序调度”。在本设计中,数据通路由 PC/NPC、寄存器堆、SEXT、ALU、流水线寄存器、存储器接口等模块构成;控制通路由控制器与冒险检测逻辑构成。

控制信号在译码阶段生成,并通过流水线寄存器传递到后续阶段。典型控制信号包括:\texttt{npc\_op}(下一 PC 选择)、\texttt{alu\_op}(ALU 运算类型)、\texttt{alub\_sel}(第二操作数选择)、\texttt{rf\_we}(寄存器写使能)、\texttt{rf\_wsel}(写回数据选择)、\texttt{ram\_we}(访存写使能)等。控制通路与数据通路的分离,使得结构更易理解,扩展也更直接。

\section{五级流水线结构}
CPU 采用经典五级流水线:取指(IF)、译码(ID)、执行(EX)、访存(MEM)、写回(WB)。通过流水线寄存器实现阶段隔离,提高吞吐率。

五级流水线的价值在于让“多个指令并行推进”。每个周期不同指令处于不同阶段,从而提升平均吞吐率。对于教学而言,这种结构也便于拆解指令执行过程,使学生明确“取指—译码—执行—访存—写回”的因果关系。

阶段划分遵循“逻辑均衡”的原则:IF 负责取指与 PC 更新,ID 负责译码与寄存器读取,EX 负责计算与分支判断,MEM 负责数据访问,WB 负责结果写回。这样每个阶段的组合逻辑深度相对均衡,更容易满足 FPGA 时序约束。

由于系统未引入缓存,IF 与 MEM 直接访问片上存储器或外设。其优点是行为可预测、调试方便;缺点是访存延迟无法隐藏,需要通过流水线控制与编译器策略保证正确性。

\figplaceholder{五级流水线结构示意图}

\section{取指与下一 PC 逻辑}
PC 模块保存当前指令地址,NPC 模块根据分支、跳转或顺序执行计算下一 PC。分支在 EX 阶段确定,控制冒险通过冲刷流水线寄存器解决。

IF 阶段除了取指外,还承担基本的对齐与地址更新工作。NPC 逻辑综合了顺序执行(PC+4)与跳转/分支目标,保证指令流的连续性。由于分支结果在 EX 阶段才能确定,因此需要在前级保留“可能错误”的指令,并在确定后冲刷。

从时序上看,PC 在时钟上升沿更新;当检测到数据冒险需要停顿时,PC 保持不变;当控制冒险确认需要跳转时,PC 直接加载 NPC 计算结果,从而“纠正”指令流。该行为保证了取指阶段与冒险控制之间的正确协同。

NPC 的计算包含三类路径:顺序执行(PC+4)、分支/跳转目标(PC+offset)以及寄存器间接跳转(rs+imm)。在本设计中,分支与 J 类指令的目标使用 \texttt{pc+offset-8} 进行修正,用于补偿流水线中已取的两条指令;而 JALR 的目标地址直接来自 \texttt{rs\_imm} 的计算结果。

复位时,PC 初始化为 0(调试模式下可能采用特殊初值以对齐 PC+4 逻辑),确保第一条指令从 IROM 起始地址获取。PC+4 既用于顺序取指,也作为 JAL/JALR 的返回地址写回,因此在 IF 阶段被提前计算并随流水线传递。

指令获取支持 IROM 与 PRAM 两个来源,通过选择信号决定“从哪一块存储器取指”。IROM 用于固化程序,PRAM 用于 Bootloader 下载后就地执行。这种双源取指机制既保证了系统启动的稳定性,又提供了快速更新程序的灵活性。

在实现上,PC 作为字节地址使用,指令存储器则以“字”为单位寻址,因此实际取指地址采用 \texttt{pc[15:2]} 或 \texttt{pc[13:2]} 的截取方式。高位地址 \texttt{pc[31:16]} 用于区分 IROM 与 PRAM 的取指区域,这使得 BIOS 与用户程序可以天然分区。

\subsection{PC/NPC 实现片段与解释}
PC 模块体现了“停顿优先于顺序推进、控制冒险优先于数据冒险”的策略:当检测到控制冒险时,PC 直接更新为 NPC;当检测到数据冒险时,PC 保持不变;否则按 NPC 正常更新。
\begin{lstlisting}[language=Verilog,caption={PC 更新逻辑(节选)}]
always @(posedge clk or posedge rst) begin
  if (rst) pc <= 32'b0;
  else if (control_hazard) pc <= npc;
  else if (data_hazard) pc <= pc;
  else pc <= npc;
end
\end{lstlisting}

NPC 逻辑负责给出“下一条指令地址”。顺序执行取 PC+4,分支/跳转取 PC+offset,寄存器间接跳转取寄存器值。在本设计中采用 \texttt{pc+offset-8} 的修正项,用于补偿流水线中已取的两条指令。
\begin{lstlisting}[language=Verilog,caption={NPC 计算逻辑(节选)}]
always @(*) begin
  if (npc_op == NPC_JMPR) npc = rs_imm;
  else if (npc_op == NPC_JMP)  npc = pc + offset - 8;
  else if (npc_op == NPC_BEQ && br == 0) npc = pc + offset - 8;
  else if (npc_op == NPC_BNE && br != 0) npc = pc + offset - 8;
  else npc = pc + 4;
end
\end{lstlisting}

\subsection{IF/ID 与 ID/EX 寄存器片段}
流水线寄存器需要支持“暂停”和“冲刷”:暂停用于数据冒险,冲刷用于控制冒险。以下片段展示了 IF/ID 与 ID/EX 在冒险发生时的处理方式。
\begin{lstlisting}[language=Verilog,caption={IF/ID 暂停与冲刷(节选)}]
always @(posedge clk or posedge rst) begin
  if (rst) ID_inst <= 32'b0;
  else if (control_hazard) ID_inst <= 32'b0;
  else if (data_hazard) ID_inst <= ID_inst;
  else ID_inst <= IF_inst;
end
\end{lstlisting}

ID/EX 对控制信号与操作数做同样的处理,确保气泡被插入到 EX 阶段。
\begin{lstlisting}[language=Verilog,caption={ID/EX 气泡插入(节选)}]
always @(posedge clk or posedge rst) begin
  if (rst) EX_rf_we <= 1'b0;
  else if (control_hazard | data_hazard) EX_rf_we <= 1'b0;
  else EX_rf_we <= ID_rf_we;
end
\end{lstlisting}

\section{译码与寄存器堆}
译码阶段解析 opcode 与 funct 字段,生成控制信号;寄存器堆支持同步写入与组合读取,满足流水线并行读取需求。

ID 阶段的关键在于把“指令位域”转换为“控制信号”。控制器根据 opcode/funct3/funct7 组合判断指令类型,并生成 ALU 操作、写回选择、立即数扩展类型等控制信号。寄存器堆的双读单写结构为执行阶段提供操作数,同时写回阶段在同一周期完成寄存器更新。

译码阶段还包含立即数生成(SEXT)逻辑。不同指令类型(I/S/B/U/J)具有不同的立即数字段拼接方式,SEXT 会根据 \texttt{sext\_op} 选择拼接规则并进行符号扩展。对于分支与跳转指令,立即数的最低位固定为 0,以满足指令对齐要求。

指令字段拆解遵循 RV32I 规范:\texttt{opcode} 位于 \texttt{inst[6:0]},\texttt{rd} 位于 \texttt{inst[11:7]},\texttt{funct3} 位于 \texttt{inst[14:12]},\texttt{rs1} 位于 \texttt{inst[19:15]},\texttt{rs2} 位于 \texttt{inst[24:20]},\texttt{funct7} 位于 \texttt{inst[31:25]}。控制器与 SEXT 都基于这些字段生成后续控制与数据路径。

寄存器堆遵循 RISC-V 约定:x0 固定为 0,读操作为组合逻辑,写操作在时钟沿更新。ID 阶段会根据指令类型判断是否需要读取源寄存器,读使能(\texttt{rf\_re})既用于节省无用读操作,也为冒险检测提供依据。

译码还会确定目的寄存器 \texttt{rd} 与是否写回。例如分支与存储指令不写回,因此 \texttt{rf\_we} 被置 0;而加载、算术与跳转类指令需要写回,写回来源由 \texttt{rf\_wsel} 指定。这样做使得“写回路径”在译码阶段就被确定,并在流水线中保持一致。

此外,ID 阶段输出的源寄存器编号、目的寄存器编号与读使能信号会被送入数据冒险检测模块,用于决定是否需要前递或停顿。通过这种“译码与冒险协作”的方式,流水线在保持高吞吐的同时避免数据错误。

\subsection{立即数扩展示意代码}
SEXT 模块根据指令类型拼接并扩展立即数,以下片段展示了不同格式的拼接方式(节选):
\begin{lstlisting}[language=Verilog,caption={SEXT 立即数扩展(节选)}]
assign ext = (sext_op == SEXT_I) ? {{20{sgn}}, din[24:13]} :
             (sext_op == SEXT_S) ? {{20{sgn}}, din[24:18], din[4:0]} :
             (sext_op == SEXT_B) ? {{19{sgn}}, din[24], din[0], din[23:18], din[4:1], 1'b0} :
             (sext_op == SEXT_U) ? {din[24:5], 12'b0} :
             (sext_op == SEXT_J) ? {{11{sgn}}, din[24], din[12:5], din[13], din[23:14], 1'b0} :
             32'b0;
\end{lstlisting}

\section{执行阶段(EX)}
执行阶段由 ALU 作为核心运算单元,完成算术逻辑运算、地址计算与分支比较。ALU 的两个操作数来自寄存器堆读出值与立即数扩展值,通过 \texttt{alub\_sel} 选择第二操作数来源。运算结果 \texttt{alu\_c} 与分支标志 \texttt{br} 被送入后续阶段。

对于算术逻辑指令,EX 直接产生最终结果;对于访存指令,EX 计算有效地址(\texttt{rs1 + imm});对于分支指令,EX 基于减法比较结果产生分支条件标志;对于 JAL/JALR,EX 阶段计算跳转目标或相关控制信号,PC+4 作为返回地址在后续阶段写回。

分支判断依赖 ALU 的比较标志。ALU 在执行减法时同时产生比较结果:相等、符号小于或符号大于。控制冒险检测逻辑依据该标志与分支类型(BEQ/BNE/BLT/BGE)决定是否跳转,从而形成“算术比较—控制决策”的闭环。

移位指令采用低 5 位作为移位量,这与 RV32I 指令规范一致。算术右移采用符号扩展,保证负数右移语义正确。

\subsection{ALU 运算与分支标志片段}
ALU 的核心是运算选择与比较标志生成。运算结果用于写回或访存地址,比较标志用于分支判断。
\begin{lstlisting}[language=Verilog,caption={ALU 运算与比较(节选)}]
B = (alub_sel == ALU_Data_Imm) ? imm : rs2;
case (alu_op)
  ALU_ADD: C_tmp = A + B;
  ALU_SUB: C_tmp = A - B;
  ALU_AND: C_tmp = A & B;
  ALU_OR : C_tmp = A | B;
  ALU_XOR: C_tmp = A ^ B;
  ALU_SLL: C_tmp = A << B[4:0];
  ALU_SRL: C_tmp = A >> B[4:0];
  ALU_SRA: C_tmp = $signed(A) >>> B[4:0];
endcase

if (alu_op != ALU_SUB) br_tmp = 0;
else if ($signed(A) == $signed(B)) br_tmp = 0;
else if ($signed(A) < $signed(B)) br_tmp = 1;
else br_tmp = 2;
\end{lstlisting}

\section{访存阶段(MEM)}
访存阶段通过 Bridge 统一访问 DRAM 或外设。若为存储指令,\texttt{Bus\_we} 置 1,地址来自 EX 计算结果,写数据来自寄存器第二操作数;若为加载指令,则从 \texttt{Bus\_rdata} 取回数据并送往写回阶段。

在外设访问时,Bridge 会将地址译码为对应外设寄存器,并返回相应读数据或执行写入动作。该机制保证了 CPU 对 DRAM 与外设的访问在语义上保持一致,软件无需区分“普通内存”与“外设寄存器”。

需要注意的是,DRAM 采用同步写入、组合读出的模型。对于紧邻的 store-load 序列,若没有适当的停顿,可能读取到旧值。为简化硬件,本项目将该类时序风险部分交由编译器处理,通过插入空操作保证数据一致性。

访存接口以 32 位数据宽度为单位,地址对齐到字边界。这样可以简化存储器接口与外设寄存器设计,但也意味着字节/半字访存不在当前硬件支持范围内。

\section{写回阶段(WB)}
写回阶段根据 \texttt{rf\_wsel} 选择写回数据来源,常见来源包括 ALU 结果、访存结果、PC+4 或立即数扩展值(如 LUI)。写回通过 \texttt{rf\_we} 控制,确保对分支/存储等不需要写回的指令关闭写使能。

寄存器堆遵循 x0 恒为 0 的规则,因此即使写回逻辑给出写地址 0,也不会改变 x0 的值。这一约束简化了软件常量的使用。

从指令角度看:算术逻辑类写回 ALU 结果;加载指令写回访存数据;JAL/JALR 写回 PC+4 作为返回地址;LUI 写回扩展后的立即数。这种“写回来源与指令类型的一一对应”让控制逻辑保持直观。

\section{流水线寄存器与阶段隔离}
流水线寄存器用于保存阶段间的“数据与控制信号快照”,保证每条指令在流水线中独立推进。以下列出各阶段寄存器保存的关键内容(概览):

\begin{longtable}{p{0.20\textwidth} p{0.72\textwidth}}
\caption{流水线寄存器关键内容(概览)}\\
\toprule
寄存器 & 保存内容(示例) \\
\midrule
\endfirsthead
\toprule
寄存器 & 保存内容(示例) \\
\midrule
\endhead
\midrule
\multicolumn{2}{r}{续下页}\\
\endfoot
\bottomrule
\endlastfoot
IF/ID & 指令字、PC+4,用于后续译码与分支处理。\\
ID/EX & 控制信号(\texttt{npc\_op}, \texttt{alu\_op}, \texttt{rf\_we} 等)与操作数(\texttt{rD1}, \texttt{rD2}, \texttt{ext})。\\
EX/MEM & 计算结果(\texttt{alu\_c})、写数据、写回控制信号。\\
MEM/WB & 访存读出数据、ALU 结果、PC+4 以及写回控制信号。\\
\end{longtable}

当出现数据冒险时,PC 与 IF/ID 保持不变,ID/EX 被清零以插入气泡;当出现控制冒险时,IF/ID 与 ID/EX 被清零以冲刷流水线。该策略简单直观,适合教学验证。

控制冒险通常具有更高优先级:一旦确认跳转成立,必须立即修正指令流,避免错误路径提交;数据冒险则通过停顿保证依赖关系。这样的优先级安排与硬件实现保持一致,也更符合直觉。

\section{关键控制信号说明}
为便于理解数据通路的“选择逻辑”,下表列出主要控制信号及其功能含义:

\begin{longtable}{p{0.25\textwidth} p{0.65\textwidth}}
\caption{控制信号与功能说明}\\
\toprule
信号 & 功能说明 \\
\midrule
\endfirsthead
\toprule
信号 & 功能说明 \\
\midrule
\endhead
\midrule
\multicolumn{2}{r}{续下页}\\
\endfoot
\bottomrule
\endlastfoot
\texttt{npc\_op} & 选择下一 PC 计算方式(PC+4、分支、跳转、寄存器间接跳转)。\\
\texttt{alu\_op} & 选择 ALU 运算类型(加减与逻辑运算等)。\\
\texttt{alub\_sel} & 选择 ALU 第二操作数来源(寄存器或立即数)。\\
\texttt{sext\_op} & 选择立即数扩展格式(I/S/B/U/J)。\\
\texttt{rf\_we} & 寄存器写回使能。\\
\texttt{rf\_wsel} & 选择写回来源(ALU、访存、PC+4、SEXT)。\\
\texttt{ram\_we} & 访存写使能(store 指令)。\\
\texttt{rf\_re} & 寄存器读使能,用于冒险检测与读端口控制。\\
\end{longtable}

\section{模块级划分与职责}
CPU 内部由多个独立模块组成,模块化设计有助于分别理解各部分功能并进行单元级调试。下表列出关键模块及其职责:

\begin{longtable}{p{0.28\textwidth} p{0.62\textwidth}}
\caption{CPU 核心模块与职责}\\
\toprule
模块 & 主要职责 \\
\midrule
\endfirsthead
\toprule
模块 & 主要职责 \\
\midrule
\endhead
\midrule
\multicolumn{2}{r}{续下页}\\
\endfoot
\bottomrule
\endlastfoot
PC & 保存并更新当前指令地址,支持停顿与跳转修正。\\
NPC & 计算下一条指令地址(PC+4、分支/跳转目标、寄存器间接跳转)。\\
IF/ID & 保存取指结果与 PC+4,支持冲刷与暂停。\\
ID/EX & 保存译码后的控制信号与操作数,支持气泡插入。\\
EX/MEM & 保存 EX 阶段计算结果与访存控制信号。\\
MEM/WB & 保存访存结果与写回控制信号。\\
RF & 32×32 寄存器堆,双读单写。\\
SEXT & 立即数字段拼接与符号扩展。\\
ALU & 算术逻辑运算与分支比较标志产生。\\
controller & 指令译码与控制信号生成。\\
Data\_Hazard\_Detection & 检测数据相关并产生前递/停顿控制。\\
Control\_Hazard\_Detection & 检测分支/跳转是否成立并触发冲刷。\\
Bridge & 地址译码与内存/外设路由。\\
\end{longtable}

\section{指令流动的周期级示例}
以一条 \texttt{add} 指令为例:第 1 个周期在 IF 取指;第 2 个周期在 ID 译码与读寄存器;第 3 个周期在 EX 完成加法;第 4 个周期在 MEM 阶段“空转”通过;第 5 个周期在 WB 写回寄存器。若后续指令与其无数据相关,则可在流水线中并行推进,实现高吞吐。

若后续指令依赖 \texttt{lw} 的结果,则在 EX 阶段检测到 load-use 冒险,PC 与 IF/ID 被暂停一个周期,ID/EX 插入气泡。通过这一机制保证正确性,同时让学生观察到“流水线性能与相关性”的直接关系。

对于分支指令(如 \texttt{beq}),指令在 EX 阶段完成比较并决定是否跳转。若跳转成立,则 IF/ID 与 ID/EX 被清空,两条已取指令被丢弃,随后从目标地址重新取指。这一过程体现了控制冒险带来的流水线“清空代价”。

\section{不同指令类型的执行路径}
为了更直观地理解控制信号与数据通路的对应关系,可将常见指令类型的执行路径概括如下:

\noindent \textbf{R 型算术逻辑}:ID 生成 \texttt{alub\_sel=reg}、\texttt{rf\_wsel=ALU}、\texttt{rf\_we=1};EX 完成运算;MEM 透传;WB 写回 ALU 结果。

\noindent \textbf{I 型算术逻辑}:与 R 型类似,但 \texttt{alub\_sel=imm},EX 使用立即数参与运算。

\noindent \textbf{加载(lw)}:EX 计算有效地址;MEM 读取数据;WB 选择访存结果写回;\texttt{rf\_we=1}。

\noindent \textbf{存储(sw)}:EX 计算有效地址;MEM 将寄存器数据写入存储器;WB 不写回(\texttt{rf\_we=0})。

\noindent \textbf{分支(beq/bne/blt/bge)}:EX 完成比较并决定是否跳转;若跳转成立则冲刷流水线;WB 不写回。

\noindent \textbf{跳转(jal/jalr)}:NPC 选择跳转目标;WB 写回 PC+4 作为返回地址;\texttt{rf\_we=1}。

\figplaceholder{数据通路与前递示意图}

\chapter{控制逻辑与冒险处理}
\section{控制信号生成}
控制器根据 opcode/funct 组合生成控制信号,决定 ALU 运算类型、寄存器写回来源、立即数扩展方式与访存控制。

从原理上看,控制器将“指令语义”映射为“数据通路选择”。例如,同一条 add 指令需要选择寄存器作为 ALU 两个输入、在 WB 阶段选择 ALU 输出写回,而 lw 指令需要选择立即数形成访存地址并在 WB 阶段选择访存数据。该过程本质上是对多路选择器与使能信号的统一调度。

\begin{lstlisting}[language=Verilog,caption={控制器片段(controller.v)}]
assign alu_op   = (inst_add | inst_addi | inst_sw | inst_lw | inst_jalr) ? `ALU_ADD :
                  (inst_and | inst_andi)                                 ? `ALU_AND :
                  (inst_or  | inst_ori)                                  ? `ALU_OR  :
                  (inst_xor | inst_xori)                                 ? `ALU_XOR :
                  (inst_sll | inst_slli)                                 ? `ALU_SLL :
                  (inst_srl | inst_srli)                                 ? `ALU_SRL :
                  (inst_sra | inst_srai)                                 ? `ALU_SRA : `ALU_SUB;
\end{lstlisting}

\section{数据冒险处理}
数据冒险通过前递与停顿两种手段解决。若后续指令依赖前一条 load 的结果,流水线将插入气泡以等待数据就绪。

前递机制能够将 EX/MEM/WB 阶段的最新结果直接送回 EX 阶段,减少不必要的停顿。只有在 load-use 情况下(数据尚未从内存返回)才需要停顿。通过这种“优先前递、必要停顿”的策略,在保持硬件复杂度可控的同时获得较高的实际吞吐率。

具体而言,数据冒险检测比较 ID 阶段的源寄存器与 EX/MEM/WB 阶段的目的寄存器,若匹配且写回使能有效,则触发前递。前递优先级通常为 EX > MEM > WB,确保使用最新结果。若 EX 阶段是加载指令(写回来源为访存数据),则即使存在匹配也无法前递数据,需要插入停顿。

在本实现中,停顿的表现为:PC 与 IF/ID 寄存器保持不变,ID/EX 寄存器被清零(插入气泡),使得冒险指令延后一周期进入 EX,从而等待数据有效。

\subsection{数据冒险检测与前递片段}
数据冒险检测主要关注“ID 阶段源寄存器是否与前级目的寄存器冲突”。以下片段展示了 load-use 停顿条件与前递选择的基本形式:
\begin{lstlisting}[language=Verilog,caption={数据冒险检测与前递(节选)}]
assign data_hazard =
  (rR1_a && EX_rf_wsel == WB_DM) ||
  (rR2_a && EX_rf_wsel == WB_DM);

always @(*) begin
  if (rR1_a)      new_rD1 = EX_alu_c;
  else if (rR1_b) new_rD1 = MEM_alu_c;
  else if (rR1_c) new_rD1 = WB_alu_c;
  else            new_rD1 = ID_rD1;
end
\end{lstlisting}

\section{控制冒险处理}
分支/跳转指令在 EX 阶段确定,IF/ID 与 ID/EX 阶段会被冲刷,确保错误路径指令不会提交。

由于未引入分支预测,控制冒险采用“确定后清空”的方式处理。这种策略简单直观,适合教学场景;代价是分支指令会带来固定的流水线气泡,但可通过实验观察到分支指令对性能的影响,从而加深理解。

具体流程为:当 EX 阶段判断分支/跳转成立时,控制冒险信号置位,PC 立即更新为 NPC,IF/ID 与 ID/EX 清零,从而丢弃已取到的错误路径指令。该机制确保控制流正确,但会产生可观测的流水线气泡,便于分析控制冒险成本。

\subsection{控制冒险检测片段}
控制冒险检测通过 EX 阶段的分支标志与指令类型判断是否需要跳转,从而触发冲刷:
\begin{lstlisting}[language=Verilog,caption={控制冒险检测(节选)}]
always @(*) begin
  if (EX_npc_op == NPC_JMPR || EX_npc_op == NPC_JMP)
    control_hazard = 1'b1;
  else if ((EX_npc_op == NPC_BEQ && alu_f == 0) ||
           (EX_npc_op == NPC_BNE && alu_f != 0) ||
           (EX_npc_op == NPC_BLT && alu_f == 1) ||
           (EX_npc_op == NPC_BGE && alu_f != 1))
    control_hazard = 1'b1;
  else
    control_hazard = 1'b0;
end
\end{lstlisting}

\figplaceholder{冒险检测与前递控制示意图}

\chapter{流水线性能与时序分析}
\section{理想吞吐与 CPI}
在理想情况下,五级流水线能够实现每周期完成一条指令的吞吐率,此时 CPI(Cycles Per Instruction)趋近于 1。然而实际系统中存在数据冒险与控制冒险,使得 CPI 大于 1。

可用如下近似关系理解性能:\\
\[
\text{CPI} \approx 1 + \text{stall\_rate} + \text{branch\_rate} \times \text{branch\_penalty}
\]
该公式虽然简化,但能帮助学生建立“冒险数量影响性能”的直观认识。

\section{分支惩罚分析}
本设计在 EX 阶段确定分支,因此在分支指令之后的若干周期可能出现气泡。若不引入分支预测,分支惩罚是固定的,这在性能上有一定代价,但结构上非常清晰,适合教学。

\section{Load-Use 冒险与停顿}
当指令依赖前一条 \texttt{lw} 的结果时,数据在 MEM 阶段才可用,必须插入一个或多个停顿周期。通过前递可以减少大部分数据冒险,但 load-use 仍然是必须停顿的典型情况。

\section{时序与关键路径}
关键路径通常出现在“寄存器读出—ALU 计算—结果选择—寄存器写回”的组合路径中。通过合理的流水线分段与寄存器隔离,可以在保证功能正确的前提下满足时序约束。本项目在 50MHz 下满足实现时序,适合教学板卡运行。

\figplaceholder{流水线时序波形示意图}

\chapter{存储系统与总线设计}
\section{IROM 与 DRAM}
指令存储器 IROM 与数据存储器 DRAM 使用 Xilinx 分布式存储器 IP 实现,深度为 16K words(约 64KB)。IROM 只读,DRAM 可读写。

为了简化设计,本项目将指令与数据存储器分离,形成哈佛风格的结构:指令通路与数据通路相互独立,便于理解流水线的取指与访存阶段。IROM 只读使得指令路径更清晰,而 DRAM 支持读写以承载变量与堆栈。

\section{PRAM 与 Bootloader}
项目包含 PRAM 作为可写程序区,BIOS 的 UART Bootloader 可将新程序写入该区域,实现在线更新。

PRAM 的引入降低了迭代成本。即使硬件不变,也可以通过串口快速更新程序内容,适合课程实验中的频繁修改与验证。

\section{总线桥与地址解码}
Bridge 模块负责将 CPU 访存地址解码为 DRAM 或外设访问请求,并将对应设备的数据返回给 CPU。

该模块是“统一访问语义”的关键:CPU 只需发出地址、读写信号与数据,而 Bridge 负责将这些请求分发至正确设备。这样,外设扩展只需在 Bridge 中新增地址范围与端口连接,而无需修改 CPU 内核。

\begin{lstlisting}[language=Verilog,caption={地址解码片段(Bridge.v)}]
wire access_mem = (addr[31:12] != 20'hFFFFF);
wire access_led = (addr == 32'hFFFF_FC60);
wire access_keypad = (addr[31:4] == 28'hFFFF_FC1);
\end{lstlisting}

\section{MMIO 地址映射}
\begin{longtable}{p{0.28\textwidth} p{0.18\textwidth} p{0.14\textwidth} p{0.32\textwidth}}
\caption{MMIO 地址映射表}\label{tab:mmio}\\
\toprule
地址范围/地址 & 设备 & 读写 & 描述 \\
\midrule
\endfirsthead
\toprule
地址范围/地址 & 设备 & 读写 & 描述 \\
\midrule
\endhead
\midrule
\multicolumn{4}{r}{续下页}\\
\endfoot
\bottomrule
\endlastfoot
0x0000\_0000--0x0000\_FFFF & DRAM & R/W & 数据存储器(64KB)\\
0x0001\_0000--0x0001\_FFFF & PRAM & R/W & 程序区(Bootloader)\\
0xFFFF\_FC00 & 数码管 & W & 32 位 BCD 显示 \\
0xFFFF\_FC10 & 键盘数据 & R/W & 键值或 0xFFFFFFFF \\
0xFFFF\_FC12 & 键盘状态 & R & bit0: 按键标志 \\
0xFFFF\_FC20 & 定时器0 & R/W & 计数器值 \\
0xFFFF\_FC24 & 定时器N & R/W & 重载值 \\
0xFFFF\_FC30 & PWM最大值 & W & 蜂鸣器频率 \\
0xFFFF\_FC34 & PWM比较值 & W & 占空比 \\
0xFFFF\_FC38 & PWM控制 & W & bit0=使能 \\
0xFFFF\_FC40 & UART数据 & R/W & 读=接收/写=发送 \\
0xFFFF\_FC44 & UART状态 & R & TX忙/RX就绪 \\
0xFFFF\_FC48 & UART控制 & W & 写 bit0 清除 RX 就绪 \\
0xFFFF\_FC50 & 看门狗 & R/W & 读计数/写喂狗 \\
0xFFFF\_FC60 & LED & W & 24 位 LED 控制 \\
0xFFFF\_FC70 & 开关 & R & 24 位开关状态 \\
0xFFFF\_FC78 & 按钮 & R & 5 位按钮状态 \\
\end{longtable}

MMIO 表将“地址空间”与“功能模块”建立了清晰对应关系。对应用开发者而言,读写特定地址即可驱动外设;对硬件开发者而言,只需保证对应地址的寄存器语义稳定,软件即可保持兼容。

\figplaceholder{MMIO 总线与外设连接示意图}

\chapter{外设控制器设计}
\section{数码管(7-Seg)}
采用动态扫描方式驱动 8 位数码管,支持 BCD 编码显示与扩展字符。显示更新频率约 10kHz,以避免闪烁。

动态扫描的原理是“分时复用”:同一时刻只点亮一位数码管,通过快速轮询形成稳定视觉效果。其实现关键在于扫描时钟与段码译码表,既要保证刷新频率,又要保证编码一致性。

\section{4×4 矩阵键盘}
键盘控制器以行列扫描方式获取按键,使用同步器与去抖逻辑提高可靠性,并将按键值锁存供软件读取。

矩阵键盘具有“行列复用”的硬件优势,但也带来去抖与锁存需求。控制器通过多周期确认与锁存机制,避免机械抖动引起的多次触发,从而保证输入稳定。

\section{LED / Switch / Button}
LED 为 24 位写寄存器控制;Switch/Button 通过读取状态寄存器获得输入。

这类外设是最简单也最直观的 I/O 实验对象。通过写 LED 与读开关/按钮,学生可以快速验证 MMIO 读写路径是否正确,并观察硬件响应与软件逻辑之间的对应关系。

\section{UART}
UART 提供基本的串口通信能力,支持阻塞读写与状态寄存器查询,用于调试与 Bootloader 传输。

UART 模块的核心是波特率分频与收发状态机。其设计强调稳定性与可预测性,使得串口既可用于交互调试,也可用于 Bootloader 的程序传输。

\section{Timer / PWM / WDT}
Timer 以计数/重载模式运行;PWM 驱动蜂鸣器输出音调;看门狗用于系统复位与稳定性保障。

Timer 为系统提供基础的时间尺度,PWM 将“计数比较”转化为占空比信号,可用于声音或电机控制;WDT 则在系统异常时触发复位,保证“可恢复性”。这些模块共同构成系统运行的基础保障。

\figplaceholder{外设控制器结构示意图}

\chapter{FPGA 实现与约束}
\section{Vivado 工程与时钟系统}
硬件工程基于 Vivado 2017.4,输入时钟为 100MHz,通过 PLL 生成 CPU 时钟,默认约 50MHz。调试模式可直接使用外部时钟。

时钟规划的目标是“稳定与可控”。将输入时钟通过 PLL 分频得到较低频率,有助于满足时序约束并降低调试难度。调试模式直连外部时钟,则便于观察波形与单步追踪。

\section{资源利用率(综合)}
根据工程自带的综合报告,核心资源占用较低,适合教学板卡。

低资源占用意味着本设计可以运行在较小的 FPGA 器件上,也为后续扩展(如加入缓存、更多外设)预留了空间。资源统计同时提供了“设计规模”的量化指标。

\begin{table}[H]
\centering
\caption{综合资源利用率(miniRV\_SoC\_utilization\_synth.rpt)}
\begin{tabular}{lrrr}
\toprule
资源类型 & Used & Available & Util\% \\
\midrule
Slice LUTs & 2286 & 63400 & 3.61 \\
Slice Registers & 2315 & 126800 & 1.83 \\
Block RAM & 0 & 135 & 0.00 \\
DSP & 0 & 240 & 0.00 \\
BUFGCTRL & 1 & 32 & 3.13 \\
Bonded IOB & 82 & 285 & 28.77 \\
\bottomrule
\end{tabular}
\end{table}

\section{时序结果(实现)}
实现报告显示设计满足时序约束,WNS 约 2.776ns,CPU 时钟约 50MHz。

从结果看,时序裕量为正,表明当前频率下路径延迟满足约束。对于教学项目,能稳定跑在 50MHz 即可满足实验需求,同时避免频率过高带来的不确定性。

\begin{table}[H]
\centering
\caption{实现时序摘要(miniRV\_SoC\_timing\_summary\_routed.rpt)}
\begin{tabular}{lrr}
\toprule
指标 & 数值 & 说明 \\
\midrule
WNS & 2.776ns & Worst Negative Slack \\
WHS & 0.067ns & Worst Hold Slack \\
CPU 时钟 & 50MHz & PLL 输出 \\
\bottomrule
\end{tabular}
\end{table}

\figplaceholder{开发板引脚与外设连接示意图}

\chapter{调试与验证方法论}
\section{分层调试思路}
调试策略建议遵循“由下到上”的分层思路:先验证时钟与复位是否正确,再验证最简单的外设(LED/数码管),再验证输入外设(按键/开关),最后验证复杂逻辑(计算器与 UART)。这种分层方法可以快速缩小问题范围,避免在未知状态下进行复杂排查。

\section{可观测点设计}
在硬件侧,可通过调试写回接口观察指令提交与寄存器写回;在软件侧,可通过 UART 输出关键状态。调试信号如 \texttt{debug\_wb\_*} 可以帮助定位“指令是否执行、是否写回正确寄存器”的问题。

\section{串口调试与日志}
UART 是教学中最常见的调试手段。通过打印关键变量、寄存器值或阶段标记,能够在不依赖波形的情况下完成基本定位。配合 BIOS 的串口函数,可快速形成调试闭环。

\section{常见问题与定位建议}
常见问题包括:MMIO 地址不一致、栈指针初始化错误、分支目标计算错误、外设时钟未正确分频等。定位时建议优先检查“地址映射—控制信号—外设响应”三条链路,逐层验证。

\figplaceholder{调试路径与观察点示意图}

\chapter{软件示例与实验验证}
\section{示例程序}
项目提供计算器、LED 波浪灯、拨码开关控制等示例,用于验证输入/输出与 BIOS API 的稳定性。

示例的设计遵循“由易到难”:先验证输出(LED/数码管),再验证输入(键盘/开关),最后验证综合逻辑(计算器)。这种渐进式的实验路径有利于快速定位问题所在。

\section{功能验证方法}
通过以下步骤进行验证:
\begin{enumerate}[label=\arabic*.]
\item 上板后进行 BIOS 自检(数码管、LED、键盘);
\item 运行计算器,测试加减乘与负数显示;
\item 使用拨码开关控制 LED,验证 MMIO 读写;
\item UART 输出日志,检查串口通信稳定性。
\end{enumerate}

\noindent 在教学实践中,建议每个示例记录“预期现象—实际现象—可能原因”,通过对比建立对系统行为的理解。这类记录不仅帮助调试,也为后续报告撰写提供依据。

\figplaceholder{示例程序运行效果照片占位}

\chapter{教学实验设计与扩展}
\section{实验目标与层次}
教学实验可分为“指令级验证”“外设级验证”“系统级验证”三层:指令级验证关注单条指令的执行语义,外设级验证关注 MMIO 读写路径,系统级验证则关注多模块协同与应用功能。

\section{推荐实验列表}
\begin{table}[H]
\centering
\caption{教学实验与目标示例}
\begin{tabular}{p{0.28\textwidth} p{0.60\textwidth}}
\toprule
实验名称 & 目标与说明 \\
\midrule
指令验证实验 & 验证 add/sub/branch 等指令执行正确性,观察寄存器写回结果。\\
LED 输出实验 & 通过循环移位控制 LED,验证写寄存器路径与时序效果。\\
键盘输入实验 & 读取矩阵键盘并在数码管显示,验证输入路径与扫描逻辑。\\
UART 调试实验 & 使用串口输出调试信息,验证通信可靠性与 BIOS 接口。\\
Bootloader 实验 & 通过串口下载程序到 PRAM,实现在线更新与执行。\\
\bottomrule
\end{tabular}
\end{table}

\section{实验扩展建议}
在完成基础实验后,可扩展如下内容:增加新的 MMIO 外设、加入简单中断机制、实现更复杂的应用程序(如计时器驱动的时钟显示)。这些扩展不仅能锻炼硬件设计能力,也能强化软件与硬件协同的理解。

\figplaceholder{教学实验流程示意图}

\chapter{总结与展望}
本项目以教学为目标,实现了从编译器到硬件的完整 RISC-V 系统。其优势在于结构清晰、模块化强、可上板验证,并提供丰富示例程序。未来可在以下方向进一步扩展:
\begin{itemize}
\item 扩展 RV32I 指令支持(如乘除、字节/半字访问);
\item 引入中断与异常机制;
\item 增加缓存与性能优化;
\item 丰富外设与图形显示模块;
\item 完善 IDE 与调试工具链。
\end{itemize}

总体而言,本项目不仅展示了“指令执行”的硬件层实现,也展示了“从高级语言到硬件执行”的完整路径。通过编译器、BIOS 与外设的配合,学生可以清晰理解软硬件协同的关键概念。后续扩展可以围绕“性能优化”“可观测性提升”和“功能完整性”三条主线展开。

\appendix

\noindent 本附录部分提供若干“查阅型”信息,便于实验与调试时快速定位:包含编译命令速查、指令清单、BIOS API 与 MMIO 地址映射。正文已给出原理说明,附录更多用于实践操作时参考。

\chapter{编译与运行命令速查}
\begin{lstlisting}[language=bash,caption={编译器与示例程序流程}]
# 编译编译器
cargo build --release

# 生成汇编与 COE
./target/release/riscv_compiler examples/bios_v2.c examples/calculator_v2.c -o output/calc_v2

# 仅生成汇编
./target/release/riscv_compiler examples/bios_v2.c examples/calculator_v2.c -S -o output/calc_v2.s
\end{lstlisting}

\chapter{指令集支持清单(详细)}
\begin{longtable}{p{0.25\textwidth} p{0.65\textwidth}}
\caption{指令分类与说明}\\
\toprule
类别 & 支持指令 \\
\midrule
\endfirsthead
\toprule
类别 & 支持指令 \\
\midrule
\endhead
\midrule
\multicolumn{2}{r}{续下页}\\
\endfoot
\bottomrule
\endlastfoot
算术/逻辑(R 型) & add, sub, and, or, xor, sll, srl, sra \\
立即数(I 型) & addi, andi, ori, xori, slli, srli, srai \\
访存(字) & lw, sw \\
分支 & beq, bne, blt, bge \\
跳转 & jal, jalr \\
高位立即数 & lui \\
\end{longtable}

\chapter{BIOS API 完整清单}
\begin{longtable}{p{0.38\textwidth} p{0.12\textwidth} p{0.12\textwidth} p{0.30\textwidth}}
\caption{BIOS API(完整版)}\\
\toprule
函数 & 参数 & 返回值 & 说明 \\
\midrule
\endfirsthead
\toprule
函数 & 参数 & 返回值 & 说明 \\
\midrule
\endhead
\midrule
\multicolumn{4}{r}{续下页}\\
\endfoot
\bottomrule
\endlastfoot
bios\_display\_bcd & int & void & 数码管显示(负号支持) \\
bios\_key\_read & - & int & 读取键盘键值 \\
bios\_led\_write & int & void & LED 写入 \\
bios\_multiply & int,int & int & 软件乘法 \\
bios\_mul10 & int & int & 乘10运算 \\
bios\_delay & int & void & 延时循环 \\
bios\_wdt\_feed & - & void & 看门狗喂狗 \\
bios\_uart\_putc & char & void & UART 发送字符 \\
bios\_uart\_puts & char* & void & UART 发送字符串 \\
bios\_uart\_putnum & int & void & UART 发送十进制 \\
bios\_uart\_puthex & int & void & UART 发送十六进制 \\
bios\_uart\_getc & - & char & UART 接收字符 \\
bios\_uart\_available & - & int & UART 是否可读 \\
bios\_buzzer\_on & - & void & 开启蜂鸣器 \\
bios\_buzzer\_off & - & void & 关闭蜂鸣器 \\
bios\_buzzer\_set & int & void & 设置 PWM 频率 \\
bios\_buzzer\_beep & int & void & 蜂鸣指定时长 \\
bios\_sw\_read & - & int & 读取拨码开关 \\
bios\_sw\_get & int & int & 读取指定拨码 \\
bios\_sw\_read\_high & - & int & 读取高 8 位开关 \\
bios\_sw\_read\_mid & - & int & 读取中 8 位开关 \\
bios\_sw\_read\_low & - & int & 读取低 8 位开关 \\
bios\_btn\_read & - & int & 读取按钮状态 \\
bios\_btn\_get & int & int & 读取指定按钮 \\
bios\_btn\_wait & - & int & 等待任意按钮按下 \\
\end{longtable}

\chapter{MMIO 地址映射(完整)}
\begin{longtable}{p{0.32\textwidth} p{0.16\textwidth} p{0.12\textwidth} p{0.32\textwidth}}
\caption{外设地址映射(完整)}\\
\toprule
地址 & 设备 & 读写 & 说明 \\
\midrule
\endfirsthead
\toprule
地址 & 设备 & 读写 & 说明 \\
\midrule
\endhead
\midrule
\multicolumn{4}{r}{续下页}\\
\endfoot
\bottomrule
\endlastfoot
0xFFFF\_FC00 & 数码管 & W & 32 位 BCD \\
0xFFFF\_FC10 & 键盘数据 & R/W & 键值/清除 \\
0xFFFF\_FC12 & 键盘状态 & R & pressed 标志 \\
0xFFFF\_FC20 & 定时器0 & R/W & 计数器 \\
0xFFFF\_FC24 & 定时器N & R/W & 重载值 \\
0xFFFF\_FC30 & PWM最大值 & W & 频率控制 \\
0xFFFF\_FC34 & PWM比较值 & W & 占空比控制 \\
0xFFFF\_FC38 & PWM控制 & W & 使能 \\
0xFFFF\_FC40 & UART数据 & R/W & 接收/发送 \\
0xFFFF\_FC44 & UART状态 & R & TX忙/RX就绪 \\
0xFFFF\_FC48 & UART控制 & W & 清除 RX 就绪 \\
0xFFFF\_FC50 & 看门狗 & R/W & 喂狗/计数 \\
0xFFFF\_FC60 & LED & W & 24 位输出 \\
0xFFFF\_FC70 & 开关 & R & 24 位输入 \\
0xFFFF\_FC78 & 按钮 & R & 5 位输入 \\
\end{longtable}

\chapter{核心模块原理摘要}
\section{miniRV\_SoC 顶层}
miniRV\_SoC 负责将 CPU、存储器与外设集成为完整 SoC,并完成时钟与复位的组织、总线互连与外设接口导出。该模块的设计重点是“解耦”:CPU 只与 Bridge 交互,Bridge 统一进行地址译码与外设路由,从而保持 CPU 核心的简洁性与可移植性。

在顶层设计中,时钟与复位的分发是稳定运行的关键。通过统一的复位域与时钟域,保证各模块同步启动,避免亚稳态与不一致状态。顶层还提供调试端口(可选),用于观察写回阶段的指令与寄存器写入,便于课程调试与波形分析。

\begin{table}[H]
\centering
\caption{miniRV\_SoC 顶层端口分组(概览)}
\begin{tabular}{p{0.25\textwidth} p{0.25\textwidth} p{0.38\textwidth}}
\toprule
端口组 & 代表信号 & 说明 \\
\midrule
时钟/复位 & \texttt{fpga\_clk}, \texttt{fpga\_rst} & 系统时钟与同步复位 \\
显示/LED & \texttt{dig\_en}, \texttt{DN\_*}, \texttt{led} & 数码管与 LED 输出 \\
输入设备 & \texttt{sw}, \texttt{button} & 拨码开关与按钮输入 \\
键盘接口 & \texttt{row}, \texttt{line} & 4×4 矩阵键盘行列线 \\
串口与蜂鸣器 & \texttt{uart\_rx}, \texttt{uart\_tx}, \texttt{buzzer} & UART 与 PWM 输出 \\
\bottomrule
\end{tabular}
\end{table}

\figplaceholder{miniRV\_SoC 连接关系示意图}

\section{myCPU 核心}
myCPU 是五级流水线 RV32I 子集处理器,实现取指、译码、执行、访存、写回的完整通路。其接口分为三类:指令获取(IROM/PRAM)、数据访存(Bridge)、以及可选的调试写回接口。通过这种划分,核心可以在不感知具体外设的情况下完成指令执行。

值得注意的是,指令获取支持 IROM 与 PRAM 两种来源,配合 Bootloader 实现“就地执行”。这种设计让程序更新更灵活,同时保持硬件结构不变。数据访问则完全通过 Bus 接口完成,CPU 不需要关心外设细节。

\begin{table}[H]
\centering
\caption{myCPU 接口分组(概览)}
\begin{tabular}{p{0.25\textwidth} p{0.28\textwidth} p{0.36\textwidth}}
\toprule
接口类型 & 代表信号 & 说明 \\
\midrule
取指接口 & \texttt{inst\_addr}, \texttt{inst\_from\_irom} & 从 IROM 取指 \\
PRAM 取指 & \texttt{inst\_addr\_dram}, \texttt{inst\_from\_dram} & Bootloader 程序区取指 \\
访存总线 & \texttt{Bus\_addr}, \texttt{Bus\_rdata}, \texttt{Bus\_we}, \texttt{Bus\_wdata} & 通过 Bridge 访问内存/外设 \\
调试接口 & \texttt{debug\_wb\_*} & 写回阶段调试信号 \\
\bottomrule
\end{tabular}
\end{table}

\figplaceholder{myCPU 内部阶段与接口示意图}

\section{controller 与 ALU}
控制器根据 opcode/funct 字段生成控制信号,决定下一 PC 选择、ALU 运算类型、立即数扩展方式与写回来源。ALU 负责算术逻辑运算并产生分支比较标志。二者的协同保证了“指令语义 → 硬件行为”的一致性。

控制器与 ALU 的解耦,使得“指令译码”和“运算执行”各自独立:控制器给出操作类型与路径选择,ALU 只负责计算。这种结构既便于扩展指令,也便于在教学中解释控制信号的作用。

\section{冒险检测与流水线寄存器}
数据冒险通过前递与暂停解决,控制冒险通过分支确定后冲刷流水线寄存器解决。流水线寄存器(IF/ID、ID/EX、EX/MEM、MEM/WB)是实现分阶段隔离的关键,使不同指令在不同阶段并行执行。

冒险检测逻辑通常需要检查“目的寄存器与源寄存器”的关系,判断是否需要前递或停顿。流水线寄存器不仅保存数据,也保存控制信号,保证各阶段的操作语义不被破坏。

\figplaceholder{流水线寄存器与冒险处理示意图}

\section{Bridge 与片上互连}
Bridge 负责地址译码,将 CPU 访问请求路由到 DRAM 或外设,同时为不同外设提供统一的读写接口。该设计将外设的地址空间解耦于 CPU,使新增外设只需扩展译码逻辑与接口连线。

此外,Bridge 会对“未命中地址”给出默认返回值,避免总线悬空。这一机制对于调试尤为重要,可快速定位非法访问或地址映射错误。

\figplaceholder{Bridge 地址译码示意图}

\chapter{外设模块原理摘要}
\section{数码管控制}
数码管采用动态扫描显示,将 8 位显示分时复用到多个段码与位选线上。驱动逻辑需满足稳定的刷新频率与段码映射规则,避免闪烁与鬼影。

在工程实现中,数码管控制通常包含一个位选计数器与段码译码器。通过轮询激活位选并输出相应段码,可以实现多位数字显示。BIOS 将应用的数值转换为 BCD,从而与显示逻辑对接。

\section{4×4 矩阵键盘}
矩阵键盘通过行列轮询检测按键。控制器依次拉低行线并读取列线状态,以确定按键位置,并配合去抖与锁存逻辑提升稳定性。

为避免按键抖动导致多次触发,控制器通常采用多周期确认与稳定计数策略。按键值锁存后由软件读取,读取完成后可通过控制寄存器清除标志。

\section{UART}
UART 模块负责串口收发。其核心是波特率分频、起始/停止位检测与收发缓冲。软件侧通过状态寄存器判断发送忙或接收就绪。

UART 的设计兼顾了“易用性”和“可视化调试”:用户可以在上位机终端直接观察输出,或通过 Bootloader 传输程序。对于教学演示,串口是最直观的调试手段之一。

\section{Timer / PWM / WDT}
Timer 提供周期计数与重载功能,可用于延时与节拍;PWM 通过比较器输出占空比信号驱动蜂鸣器;WDT 通过计数与喂狗机制保证系统在异常时复位。

Timer 与 PWM 的组合可以实现“节拍 + 输出”类实验,例如蜂鸣器发声或 LED 闪烁。WDT 则更偏向系统可靠性演示,体现嵌入式系统的基本安全机制。

\section{LED / Switch / Button}
LED 为写寄存器驱动,开关与按钮为读寄存器输入。该类外设强调简洁可用,方便进行基础 I/O 实验验证。

通过这些最基础的外设,学生可以建立对“硬件寄存器—软件访问”的直观认识,为理解更复杂外设打下基础。

\figplaceholder{外设控制逻辑示意图}

\chapter{编译器模块原理摘要}
\section{前端:词法与语法}
词法分析将字符流转为 Token 流,语法分析依据语法规则构建 AST。该过程的目标是将线性文本转换为结构化语法树,为后续语义检查与代码生成提供基础。

在实现上,词法阶段处理关键字、标识符、数字与运算符;语法阶段根据优先级和结合性解析表达式,并构建语句块与函数结构。这样得到的 AST 更接近语言语义,而非单纯的文本结构。

\section{语义分析与符号表}
语义分析阶段维护符号表与作用域信息,确保变量/函数的声明与使用一致,同时完成类型检查与简单错误诊断。

符号表不仅记录变量位置,也记录类型与栈偏移,便于后续代码生成阶段直接查询。对于教学而言,这一过程可以帮助学生理解“变量名到内存位置”的映射关系。

\section{代码生成与栈帧管理}
代码生成将 AST 映射为 RV32I 指令序列,并依据简化调用约定分配参数寄存器与栈空间。当前实现采用固定大小栈帧并保存返回地址,便于教学与调试。

固定栈帧的优势是实现简单、行为可预测,便于在波形或仿真中观察栈指针变化。虽然牺牲了部分空间效率,但更适合教学目标。

\section{汇编与链接}
汇编器将汇编文本解析为目标格式,链接器负责段布局与重定位,最终输出 COE 文件用于 FPGA 初始化。

由于目标平台是 FPGA,本项目将“链接后的机器码”直接转换为 COE 文件,这使得软件代码可以像硬件配置一样被加载,体现“软硬一体化”的特征。

\figplaceholder{编译器各阶段输入输出示意图}

\chapter{示例程序说明}
\section{计算器}
计算器示例通过键盘输入数字与运算符,调用 BIOS 进行数码管显示与乘法运算。其核心流程为:输入解析 → 运算选择 → 结果计算 → 显示输出。

在教学实践中,该示例可用于讲解“键盘编码—软件解析—数值运算—显示编码”的完整链路,尤其适合演示系统各模块之间的协作。

\section{LED 波浪灯}
LED 波浪灯通过移位与延时产生动态光效,验证基本输出与延时机制,适用于观察系统运行是否稳定。

该示例强调时序与视觉效果的对应关系,便于理解“软件循环 + 硬件输出”的实时性。

\section{拨码开关控制}
通过读取拨码开关状态并映射到 LED,实现“输入—输出”闭环验证,可用于测试 MMIO 读写正确性。

该实验同时验证了“读外设—处理—写外设”的最小闭环,适合快速排查总线与地址映射问题。

\chapter{汇编与 COE 输出结构说明}
\section{汇编输出组织}
汇编输出通常包含 \texttt{.text} 段指令与若干初始化伪指令。编译器负责生成函数入口、栈帧建立与返回指令,保证程序可执行。

在教学中,可以通过阅读汇编输出理解调用约定与栈帧布局,例如栈指针调整、返回地址保存与恢复等。这些内容直接体现了“高级语言语义如何落实到指令层”。

\section{COE 格式要点}
COE 文件是 Vivado 内存初始化格式,以十六进制向量表示机器码。其核心要素是 radix 与 vector 两个字段。

COE 的优势在于“工具链友好”:Vivado 可直接读取并初始化存储器,从而在硬件层面完成软件加载。对于教学实验,COE 相当于“硬件可识别的软件镜像”。

\begin{lstlisting}[caption={COE 文件头示意}]
memory_initialization_radix=16;
memory_initialization_vector=
00008137,
00010113,
FFFFF2B7,
...\end{lstlisting}

\end{document}
